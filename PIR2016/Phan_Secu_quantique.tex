
\documentclass[12pt]{Sciences}
\UnivYear{2016-2017}

%\usepackage[T1]{fontenc}
%\usepackage{amsmath}
%\usepackage[frenchb]{babel}


\usepackage[utf8]{inputenc}
\usepackage{hyperref}
\usepackage{color}
\usepackage{graphicx,amsmath,enumerate}
\graphicspath{{figs/}}

\newcommand{\NP}{{\em NP\,}}
\pagestyle{empty}


%%%%%%%%%% Start TeXmacs macros
%\usepackage[french]{babel}
\newcommand{\tmstrong}[1]{\textbf{#1}}
\newenvironment{enumeratenumeric}{\begin{enumerate}[1.] }{\end{enumerate}}
%\newtheorem{definition}{D\'efinition}
%{\theorembodyfont{\rmfamily\small}\newtheorem{exercise}{Exercice}}
%%%%%%%%%% End TeXmacs macros



\title{PIR : La sécurité contre les attaques quantiques\\ 
\emph{PHAN Duong Hieu }}
\begin{document}

\maketitle

\thispagestyle{empty}

\paragraph{Résumé.}
La mise en œuvre
éventuelle de machines quantiques rendrait plusieurs schémas
cryptographiques vulnérables. En effet, de nombreux problèmes
peuvent être résolus en temps polynomial par des machines quantiques comme la
factorisation et le logarithme discret (qui
font partie d’une classe de problèmes dits de sous-
groupe caché dans
des groupes abéliens) qui sont largement utilisés dans des systèmes pratiques. Il est par
conséquent utile d’étudier des constructions basées sur des
problèmes algorithmiques qui
sont considérés difficiles à résoudre par
des machines quantiques. Les problèmes
algorithmiques demeurant non
résolus par les machines quantiques sont à titre d’exemple :
décodage des codes linéaires ; la recherche du vecteur le plus court
dans un réseau
(lattice); le problème de sous-groupe caché dans des
groupes non abéliens (l’isomorphisme
de graphes par exemple), etc. Des
constructions de schémas cryptographiques ayant
comme base ces
problèmes sont envisageables. Dans notre projet de stage, nous nous
intéressons aux problèmes de la théorie des codes et aux problèmes de
réseaux. 


\paragraph{Référence.}
\begin{itemize}
\item A Decade of Lattice Cryptography\\
\url{https://web.eecs.umich.edu/~cpeikert/pubs/lattice-survey.pdf}
\item Hardness of k-LWE and Applications in Traitor Tracing
\url{http://www.di.ens.fr/users/phan/2016_LBTTlong.pdf}

\end{itemize}

\end{document}
\documentclass[12pt]{Sciences}
\UnivYear{2016-2017}

%\usepackage[T1]{fontenc}
%\usepackage{amsmath}
%\usepackage[frenchb]{babel}

\usepackage[utf8]{inputenc}


\usepackage{color}
\usepackage{graphicx,amsmath,enumerate}
\graphicspath{{figs/}}

\newcommand{\NP}{{\em NP\,}}
\pagestyle{empty}


%%%%%%%%%% Start TeXmacs macros
%\usepackage[french]{babel}
\newcommand{\tmstrong}[1]{\textbf{#1}}
\newenvironment{enumeratenumeric}{\begin{enumerate}[1.] }{\end{enumerate}}
%\newtheorem{definition}{D\'efinition}
%{\theorembodyfont{\rmfamily\small}\newtheorem{exercise}{Exercice}}
%%%%%%%%%% End TeXmacs macros



\title{PIR : La sécurité contre les attaques quantiques\\ 
\emph{PHAN Duong Hieu }}
\begin{document}

\maketitle

\thispagestyle{empty}

\paragraph{Résumé.}


\end{document}


