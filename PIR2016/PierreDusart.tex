\documentclass[12pt,a4paper]{article}
%\usepackage[applemac]{inputenc}
%\usepackage[T1]{fontenc}
%\usepackage[latin1]{inputenc}
%\usepackage[francais]{babel} 

\usepackage[utf8]{inputenc}
\pagestyle{empty}

\usepackage{makeidx}               % include an index at the end
\makeindex
\usepackage{color}
\usepackage{graphicx}              % for \includegraphics
\usepackage{hyperref}              % for HyperTeX links
\everymath{\displaystyle}
\usepackage{amsmath,amsfonts,amssymb,amsthm}
\usepackage{mathpple}
\usepackage[notcite]{showkeys}

\def\dvilink#1#2{\mbox{\href{dvi:#1}{\verb{#2}}}}


\newcommand{\RR}{\mathbb R}
\newcommand{\F}{\mathbb F}


\def\titre#1{\noindent{$\star$ \bf #1} \\ }
\def\auteur#1{\noindent{\large{ \em #1}}  }
\newcommand{\resume}{\noindent{\textbf{Résumé~: }}  }
\newcommand{\prerequis}{\noindent{\textbf{Prérequis~: }}  }
\newcommand{\reference}{\noindent{\textbf{Références~: }}  }

\begin{document}

\auteur{Pierre Dusart}\titre{Nombres premiers jumeaux}
\resume Ce projet consiste \`a analyser les r\'esultats de Marc \& Fran\c{c}ois WOLF qui d\'eveloppent une th\'eorie \`a base d'indices sur les nombres impairs pour donner des formules explicites reliant les couples de nombres premiers jumeaux.
 \\
\prerequis connaissances de base sur les nombres premiers.
\\
\reference \verb+www.mathscience.tsoftemail.com+,
    {Octobre 2016}.

\end{document}