\documentclass[12pt,a4paper]{article}
\usepackage[utf8]{inputenc}
%\usepackage[T1]{fontenc}
%\usepackage[latin1]{inputenc}
\usepackage[francais]{babel} 
\pagestyle{empty}

\usepackage{makeidx}               % include an index at the end
\makeindex
\usepackage{color}
\usepackage{graphicx}              % for \includegraphics
\usepackage{hyperref}              % for HyperTeX links
\everymath{\displaystyle}
\usepackage{amsmath,amsfonts,amssymb,amsthm}
\usepackage{mathpple}
\usepackage[notcite]{showkeys}

%\def\dvilink#1#2{\mbox{\href{dvi:#1}{\verb{#2}}}}
%
%
%\newcommand{\RR}{\mathbb R}
%\newcommand{\F}{\mathbb F}
%
%\def\titre#1{\noindent{$\star$ \bf #1} \\ }
%\def\auteur#1{\noindent{\large{ \em #1}}  }
%\newcommand{\resume}{\noindent{\textbf{Résumé~: }}  }
%\newcommand{\prerequis}{\noindent{\textbf{Prérequis~: }}  }
%\newcommand{\reference}{\noindent{\textbf{Références~: }}  }

\begin{document}


\noindent{$\star$ \textbf{Algèbre linéaire : inversion et précision}
\\
\noindent \textbf{Prérequis~: } (algèbre linéaire) 

Effectuer efficacement les opérations 
de l'algèbre linéaire est un problème crucial : il y a peu d'applications mathématiques
effectives qui ne reposent pas à un moment ou à un autre sur de l'algèbre linéaire !
Nous allons nous intéresser ici au problème, particulièrement fondamental, de l'inversion d'une matrice. 
 \\
\noindent \textbf{Résumé~: }  Il existe beaucoup de méthodes et d'algorithmes pour 
calculer l'inverse d'une matrice : algorithme de Gauss, méthodes sans division, méthodes numériques ou de calcul formel $\dots$
Nous souhaitons comparer trois de ces méthodes pour le calcul de l'inverse d'une matrice à coefficients rationnels :
\begin{enumerate}
\item utiliser un algorithme de Gauss en faisant du calcul formel ;
\item utiliser une méthode de type symbolique-numérique en passant par les flottants ;
\item utiliser une méthode de type symbolique-numérique $p$-adique.
\end{enumerate}
Dans chaque cas, on s'intéressera à l'estimation du temps de calcul et à la gestion de la précision.
Nous aimerions répondre aux questions suivantes : 
\begin{itemize}
\item Comment comparer ces méthodes ?
\item Quelle méthode pour quelles matrices ?
\end{itemize}
\par

\noindent \textbf{Références~: }  

Gene H. GOLUB, Charles F. VAN LOAN.
	{\em{Matrix Computations}};
	
Joachim von zur GATHEN, Jürgen GERHARD.  
	{\em{Modern Computer Algebra}};
	
Tristan VACCON.
	{\em{Précision $p$-adique}};
	
\end{document}