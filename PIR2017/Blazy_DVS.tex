\documentclass[12pt,a4paper]{article}
%\usepackage[applemac]{inputenc}
%\usepackage[T1]{fontenc}
%\usepackage[latin1]{inputenc}
%\usepackage[francais]{babel} 

\usepackage[utf8]{inputenc}

\pagestyle{empty}

\usepackage{makeidx}               % include an index at the end
\makeindex
\usepackage{color}
\usepackage{graphicx}              % for \includegraphics
\usepackage{hyperref}              % for HyperTeX links
\everymath{\displaystyle}
\usepackage{amsmath,amsfonts,amssymb,amsthm}
\usepackage{mathpple}
\usepackage[notcite]{showkeys}

\def\dvilink#1#2{\mbox{\href{dvi:#1}{\verb{#2}}}}


\newcommand{\RR}{\mathbb R}
\newcommand{\F}{\mathbb F}

\def\titre#1{\noindent{$\star$ \bf #1} \\ }
\def\auteur#1{\noindent{\large{ \em #1}}  }
\newcommand{\resume}{\noindent{\textbf{Résumé~: }}  }
\newcommand{\prerequis}{\noindent{\textbf{Prérequis~: }}  }
\newcommand{\reference}{\noindent{\textbf{Références~: }}  }

\begin{document}

\auteur{Olivier Blazy}
\titre{Designated Verifier Signature}
\resume 
 \\Les signatures permettent d'authentifier un message (mail) pour que le
destinataire soit convaincu de sa provenance. Cependant dans certains
cas, on veut pouvoir faire en sorte que le destinataire ne puisse pas
convaincre quelqu'un d'autre que le message vient bien de nous.

Le sujet consiste à regarder les diverses solutions théoriques
proposées, et essayer de les catégoriser pour dégager les grands
mécanismes cryptographique utilisés pour y arriver.
%\prerequis aucun.
\\ 
%\reference 
\end{document}