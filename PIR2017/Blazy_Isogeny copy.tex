\documentclass[12pt,a4paper]{article}
%\usepackage[applemac]{inputenc}
%\usepackage[T1]{fontenc}
%\usepackage[latin1]{inputenc}
%\usepackage[francais]{babel} 

\usepackage[utf8]{inputenc}

\pagestyle{empty}

\usepackage{makeidx}               % include an index at the end
\makeindex
\usepackage{color}
\usepackage{graphicx}              % for \includegraphics
\usepackage{hyperref}              % for HyperTeX links
\everymath{\displaystyle}
\usepackage{amsmath,amsfonts,amssymb,amsthm}
\usepackage{mathpple}
\usepackage[notcite]{showkeys}

\def\dvilink#1#2{\mbox{\href{dvi:#1}{\verb{#2}}}}


\newcommand{\RR}{\mathbb R}
\newcommand{\F}{\mathbb F}

\def\titre#1{\noindent{$\star$ \bf #1} \\ }
\def\auteur#1{\noindent{\large{ \em #1}}  }
\newcommand{\resume}{\noindent{\textbf{Résumé~: }}  }
\newcommand{\prerequis}{\noindent{\textbf{Prérequis~: }}  }
\newcommand{\reference}{\noindent{\textbf{Références~: }}  }

\begin{document}

\auteur{Olivier Blazy}
\titre{Supersingular Isogenies}
\resume 
 \\ Les isognénies supersingulières sont un sujet actif de recherche en
cryptographie post-quantique. Un échange de clé sécurisé ainsi qu'un
schéma de chiffremenet efficaces ont déjà vu le jour.
Le but de ce projet est de comprendre et maitriser les notions autour de
ce concept, en expliquer les enjeux, les difficultés.
%\prerequis aucun.
\\ 
%\reference 

\paragraph{Référence}
\begin{itemize}
\item \url{https://arxiv.org/pdf/1711.04062.pdf}
\end{itemize}
\end{document}