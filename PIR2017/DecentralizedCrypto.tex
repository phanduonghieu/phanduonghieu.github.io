\documentclass[12pt,a4paper]{article}
%\usepackage[applemac]{inputenc}
%\usepackage[T1]{fontenc}
%\usepackage[latin1]{inputenc}
%\usepackage[francais]{babel} 

\usepackage[utf8]{inputenc}

\pagestyle{empty}

\usepackage{makeidx}               % include an index at the end
\makeindex
\usepackage{color}
\usepackage{graphicx}              % for \includegraphics
\usepackage{hyperref}              % for HyperTeX links
\everymath{\displaystyle}
\usepackage{amsmath,amsfonts,amssymb,amsthm}
\usepackage{mathpple}
\usepackage[notcite]{showkeys}

\def\dvilink#1#2{\mbox{\href{dvi:#1}{\verb{#2}}}}


\newcommand{\RR}{\mathbb R}
\newcommand{\F}{\mathbb F}

\def\titre#1{\noindent{$\star$ \bf #1} \\ }
\def\auteur#1{\noindent{\large{ \em #1}}  }
\newcommand{\resume}{\noindent{\textbf{Résumé~: }}  }
\newcommand{\prerequis}{\noindent{\textbf{Prérequis~: }}  }
\newcommand{\reference}{\noindent{\textbf{Références~: }}  }

\begin{document}

\auteur{Duong Hieu PHAN}
\titre{Decentralized Cryptography}
\resume 
Un axe de recherche très actif en ce moment est d'étudier la décentralization des schémas cryptographiques où aucune confiance en autorités n'est demandé et chaque utilisateur contribue à la génération des parametrès du système. Blockchain est une méthode intéressante pour décentralizer la validation des transactions mais cette technique ne peut pas être appliquée à des objectif plus avancés, à nommer les calculs distribués. L'objectif de ce projet est d'étudier des méthodes de décentralizer les calculs entre plusieurs acteurs. 
\prerequis aucun.
\\ 
\reference 
\begin{itemize}
\item Decentralized Dynamic Broadcast Encryption\\
Duong Hieu Phan, David Pointcheval and Mario Strefler
\url{https://www.di.ens.fr/users/phan/2012_scn.pdf}
\item Decentralized Multi-Client Functional Encryption for Inner Product \\
Jérémy Chotard, Edouard Dufour Sans, Romain Gay, Duong Hieu Phan and David Pointcheval.
\url{https://eprint.iacr.org/2017/989.pdf}
\end{itemize}
	
\end{document}