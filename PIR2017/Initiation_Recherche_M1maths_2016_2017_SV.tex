\documentclass[12pt,a4paper]{article}
\usepackage[applemac]{inputenc}
%\usepackage[T1]{fontenc}
%\usepackage[latin1]{inputenc}
\usepackage[francais]{babel} 
\pagestyle{empty}

\usepackage{makeidx}               % include an index at the end
\makeindex
\usepackage{color}
\usepackage{graphicx}              % for \includegraphics
\usepackage{hyperref}              % for HyperTeX links
\everymath{\displaystyle}
\usepackage{amsmath,amsfonts,amssymb,amsthm}
\usepackage{mathpple}
\usepackage[notcite]{showkeys}

\def\dvilink#1#2{\mbox{\href{dvi:#1}{\verb{#2}}}}


\newcommand{\RR}{\mathbb R}
\newcommand{\F}{\mathbb F}

\def\titre#1{\noindent{$\star$ \bf #1} \\ }
\def\auteur#1{\noindent{\large{ \em #1}}  }
\newcommand{\resume}{\noindent{\textbf{R�sum�~: }}  }
\newcommand{\prerequis}{\noindent{\textbf{Pr�requis~: }}  }
\newcommand{\reference}{\noindent{\textbf{R�f�rences~: }}  }

\begin{document}

\auteur{St�phane Vinatier}�\titre{Le probl�me de Kakeya pour les corps finis}
\resume (combinatoire, alg�bre) Soit $p$ un nombre premier et ${\mathbb F}_p$ le corps fini � $p$ �l�ments. Le but de ce projet est d'�tudier le cardinal des plus petites parties du plan ${{\mathbb F}_p}^2$ qui contiennent une droite dans chaque direction possible. On se basera sur l'article de 10 pages (en anglais) cit� en r�f�rence, qui donne des r�sultats et des conjectures autour de cette question.
 \\
\prerequis connaissances de base sur les corps finis.
\\ 
\reference  Faber X. W. C., On the finite field Kakeya problem in two dimensions, {\it J. Number Theory} {\bf 124} (2007), no. 1, 248�257.
	
\end{document}
