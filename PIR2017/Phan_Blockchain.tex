\documentclass[12pt,a4paper]{article}
%\usepackage[applemac]{inputenc}
%\usepackage[T1]{fontenc}
%\usepackage[latin1]{inputenc}
%\usepackage[francais]{babel} 

\usepackage[utf8]{inputenc}

\pagestyle{empty}

\usepackage{makeidx}               % include an index at the end
\makeindex
\usepackage{color}
\usepackage{graphicx}              % for \includegraphics
\usepackage{hyperref}              % for HyperTeX links
\everymath{\displaystyle}
\usepackage{amsmath,amsfonts,amssymb,amsthm}
\usepackage{mathpple}
\usepackage[notcite]{showkeys}

\def\dvilink#1#2{\mbox{\href{dvi:#1}{\verb{#2}}}}


\newcommand{\RR}{\mathbb R}
\newcommand{\F}{\mathbb F}

\def\titre#1{\noindent{$\star$ \bf #1} \\ }
\def\auteur#1{\noindent{\large{ \em #1}}  }
\newcommand{\resume}{\noindent{\textbf{Résumé~: }}  }
\newcommand{\prerequis}{\noindent{\textbf{Prérequis~: }}  }
\newcommand{\reference}{\noindent{\textbf{Références~: }}  }

\begin{document}

\auteur{Duong Hieu PHAN}
\titre{Blockchain et des applications}
\resume 
 \\ 
Ce sujet est assez large et on peut décider de choisir une direction concrète. Dans un premier temps, vous étudiez la technologie "blockchain" et ses applications. On essaie ensuite d'identifier des limitations de ce model et de comprendre des solutions proposées. 
%\prerequis aucun.
\\ 
%\reference 

\paragraph{Quelques exemples de référence}
\begin{itemize}
\item The Bitcoin Backbone Protocol: Analysis and Applications\\
\url{https://eprint.iacr.org/2014/765}
\item The Bitcoin Lightning Network:
Scalable Off-Chain Instant Payments
\url{https://lightning.network/lightning-network-paper.pdf}
\end{itemize}
\end{document}